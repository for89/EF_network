\documentclass[
  man,
  floatsintext,
  longtable,
  nolmodern,
  notxfonts,
  notimes,
  colorlinks=true,linkcolor=blue,citecolor=blue,urlcolor=blue]{apa7}

\usepackage{amsmath}
\usepackage{amssymb}




\RequirePackage{longtable}
\RequirePackage{threeparttablex}

\makeatletter
\renewcommand{\paragraph}{\@startsection{paragraph}{4}{\parindent}%
	{0\baselineskip \@plus 0.2ex \@minus 0.2ex}%
	{-.5em}%
	{\normalfont\normalsize\bfseries\typesectitle}}

\renewcommand{\subparagraph}[1]{\@startsection{subparagraph}{5}{0.5em}%
	{0\baselineskip \@plus 0.2ex \@minus 0.2ex}%
	{-\z@\relax}%
	{\normalfont\normalsize\bfseries\itshape\hspace{\parindent}{#1}\textit{\addperi}}{\relax}}
\makeatother




\usepackage{longtable, booktabs, multirow, multicol, colortbl, hhline, caption, array, float, xpatch}
\usepackage{subcaption}


\renewcommand\thesubfigure{\Alph{subfigure}}
\setcounter{topnumber}{2}
\setcounter{bottomnumber}{2}
\setcounter{totalnumber}{4}
\renewcommand{\topfraction}{0.85}
\renewcommand{\bottomfraction}{0.85}
\renewcommand{\textfraction}{0.15}
\renewcommand{\floatpagefraction}{0.7}

\usepackage{tcolorbox}
\tcbuselibrary{listings,theorems, breakable, skins}
\usepackage{fontawesome5}

\definecolor{quarto-callout-color}{HTML}{909090}
\definecolor{quarto-callout-note-color}{HTML}{0758E5}
\definecolor{quarto-callout-important-color}{HTML}{CC1914}
\definecolor{quarto-callout-warning-color}{HTML}{EB9113}
\definecolor{quarto-callout-tip-color}{HTML}{00A047}
\definecolor{quarto-callout-caution-color}{HTML}{FC5300}
\definecolor{quarto-callout-color-frame}{HTML}{ACACAC}
\definecolor{quarto-callout-note-color-frame}{HTML}{4582EC}
\definecolor{quarto-callout-important-color-frame}{HTML}{D9534F}
\definecolor{quarto-callout-warning-color-frame}{HTML}{F0AD4E}
\definecolor{quarto-callout-tip-color-frame}{HTML}{02B875}
\definecolor{quarto-callout-caution-color-frame}{HTML}{FD7E14}

%\newlength\Oldarrayrulewidth
%\newlength\Oldtabcolsep


\usepackage{hyperref}




\providecommand{\tightlist}{%
  \setlength{\itemsep}{0pt}\setlength{\parskip}{0pt}}
\usepackage{longtable,booktabs,array}
\usepackage{calc} % for calculating minipage widths
% Correct order of tables after \paragraph or \subparagraph
\usepackage{etoolbox}
\makeatletter
\patchcmd\longtable{\par}{\if@noskipsec\mbox{}\fi\par}{}{}
\makeatother
% Allow footnotes in longtable head/foot
\IfFileExists{footnotehyper.sty}{\usepackage{footnotehyper}}{\usepackage{footnote}}
\makesavenoteenv{longtable}

\usepackage{graphicx}
\makeatletter
\newsavebox\pandoc@box
\newcommand*\pandocbounded[1]{% scales image to fit in text height/width
  \sbox\pandoc@box{#1}%
  \Gscale@div\@tempa{\textheight}{\dimexpr\ht\pandoc@box+\dp\pandoc@box\relax}%
  \Gscale@div\@tempb{\linewidth}{\wd\pandoc@box}%
  \ifdim\@tempb\p@<\@tempa\p@\let\@tempa\@tempb\fi% select the smaller of both
  \ifdim\@tempa\p@<\p@\scalebox{\@tempa}{\usebox\pandoc@box}%
  \else\usebox{\pandoc@box}%
  \fi%
}
% Set default figure placement to htbp
\def\fps@figure{htbp}
\makeatother


% definitions for citeproc citations
\NewDocumentCommand\citeproctext{}{}
\NewDocumentCommand\citeproc{mm}{%
  \begingroup\def\citeproctext{#2}\cite{#1}\endgroup}
\makeatletter
 % allow citations to break across lines
 \let\@cite@ofmt\@firstofone
 % avoid brackets around text for \cite:
 \def\@biblabel#1{}
 \def\@cite#1#2{{#1\if@tempswa , #2\fi}}
\makeatother
\newlength{\cslhangindent}
\setlength{\cslhangindent}{1.5em}
\newlength{\csllabelwidth}
\setlength{\csllabelwidth}{3em}
\newenvironment{CSLReferences}[2] % #1 hanging-indent, #2 entry-spacing
 {\begin{list}{}{%
  \setlength{\itemindent}{0pt}
  \setlength{\leftmargin}{0pt}
  \setlength{\parsep}{0pt}
  % turn on hanging indent if param 1 is 1
  \ifodd #1
   \setlength{\leftmargin}{\cslhangindent}
   \setlength{\itemindent}{-1\cslhangindent}
  \fi
  % set entry spacing
  \setlength{\itemsep}{#2\baselineskip}}}
 {\end{list}}
\usepackage{calc}
\newcommand{\CSLBlock}[1]{\hfill\break\parbox[t]{\linewidth}{\strut\ignorespaces#1\strut}}
\newcommand{\CSLLeftMargin}[1]{\parbox[t]{\csllabelwidth}{\strut#1\strut}}
\newcommand{\CSLRightInline}[1]{\parbox[t]{\linewidth - \csllabelwidth}{\strut#1\strut}}
\newcommand{\CSLIndent}[1]{\hspace{\cslhangindent}#1}


\usepackage[nolongtablepatch]{lineno}
\linenumbers



\usepackage{newtx}

\defaultfontfeatures{Scale=MatchLowercase}
\defaultfontfeatures[\rmfamily]{Ligatures=TeX,Scale=1}





\title{Mapping the Structure of Executive Function in Early Childhood
using Network Analysis}


\shorttitle{Mapping Executive Function}


\usepackage{etoolbox}









\authorsnames[{1,2},{2},{3},{4},{5},{3},{2},{2}]{Fionnuala
O'Reilly,Jelena Sucevic,Caylee Cook,Justin E. Karr,Philippe
Rast,Catherine Draper*,Steven Howard*,Gaia Scerif*}







\authorsaffiliations{
{Stanford University},{University of Oxford},{University of the
Witwatersrand},{University of Kentucky},{University of California,
Davis}}




\leftheader{O'Reilly, Sucevic, Cook, Karr, Rast, Draper*, Howard* and Scerif*}



\abstract{TO BE ADDED. }

\keywords{executive function}

\authornote{\par{\addORCIDlink{Fionnuala
O'Reilly}{0000-0002-4355-9088}}\par{\addORCIDlink{Jelena
Sucevic}{0000-0001-5091-5434}}\par{\addORCIDlink{Caylee
Cook}{0000-0001-9718-8887}}\par{\addORCIDlink{Justin E.
Karr}{0000-0003-3653-332X}}\par{\addORCIDlink{Philippe
Rast}{0000-0003-3630-6629}}\par{\addORCIDlink{Catherine
Draper*}{0000-0002-2885-437X}}\par{\addORCIDlink{Steven
Howard*}{0000-0002-1258-3210}}\par{\addORCIDlink{Gaia
Scerif*}{0000-0002-6371-8874}} 

\par{   The author has no conflict of interest to declare. This work was
supported by X.   }
\par{Correspondence concerning this article should be addressed
to Fionnuala
O'Reilly, Email: \href{mailto:fionnuala.oreilly@psy.ox.ac.uk}{fionnuala.oreilly@psy.ox.ac.uk}}
}

\makeatletter
\let\endoldlt\endlongtable
\def\endlongtable{
\hline
\endoldlt
}
\makeatother

\urlstyle{same}



\usepackage{setspace}
% \AtBeginEnvironment{longtable}{\singlespacing\footnotesize}
\AtBeginEnvironment{longtable}{\singlespacing\renewcommand*{\arraystretch}{1.15}}
\usepackage{booktabs}
\usepackage{caption}
\usepackage{longtable}
\usepackage{colortbl}
\usepackage{array}
\usepackage{anyfontsize}
\usepackage{multirow}
\makeatletter
\@ifpackageloaded{caption}{}{\usepackage{caption}}
\AtBeginDocument{%
\ifdefined\contentsname
  \renewcommand*\contentsname{Table of contents}
\else
  \newcommand\contentsname{Table of contents}
\fi
\ifdefined\listfigurename
  \renewcommand*\listfigurename{List of Figures}
\else
  \newcommand\listfigurename{List of Figures}
\fi
\ifdefined\listtablename
  \renewcommand*\listtablename{List of Tables}
\else
  \newcommand\listtablename{List of Tables}
\fi
\ifdefined\figurename
  \renewcommand*\figurename{Figure}
\else
  \newcommand\figurename{Figure}
\fi
\ifdefined\tablename
  \renewcommand*\tablename{Table}
\else
  \newcommand\tablename{Table}
\fi
}
\@ifpackageloaded{float}{}{\usepackage{float}}
\floatstyle{ruled}
\@ifundefined{c@chapter}{\newfloat{codelisting}{h}{lop}}{\newfloat{codelisting}{h}{lop}[chapter]}
\floatname{codelisting}{Listing}
\newcommand*\listoflistings{\listof{codelisting}{List of Listings}}
\makeatother
\makeatletter
\usepackage{pdflscape}
\makeatother
\makeatletter
\makeatother
\makeatletter
\@ifpackageloaded{caption}{}{\usepackage{caption}}
\@ifpackageloaded{subcaption}{}{\usepackage{subcaption}}
\makeatother

% From https://tex.stackexchange.com/a/645996/211326
%%% apa7 doesn't want to add appendix section titles in the toc
%%% let's make it do it
\makeatletter
\xpatchcmd{\appendix}
  {\par}
  {\addcontentsline{toc}{section}{\@currentlabelname}\par}
  {}{}
\makeatother

%% Disable longtable counter
%% https://tex.stackexchange.com/a/248395/211326

\usepackage{etoolbox}

\makeatletter
\patchcmd{\LT@caption}
  {\bgroup}
  {\bgroup\global\LTpatch@captiontrue}
  {}{}
\patchcmd{\longtable}
  {\par}
  {\par\global\LTpatch@captionfalse}
  {}{}
\apptocmd{\endlongtable}
  {\ifLTpatch@caption\else\addtocounter{table}{-1}\fi}
  {}{}
\newif\ifLTpatch@caption
\makeatother

\begin{document}

\maketitle




\setlength\LTleft{0pt}

\resetlinenumber[1]



\section{Introduction}\label{introduction}

\subsection{Research Questions}\label{research-questions}

\textbf{RQ1 (Primary): EF Network Structure Within Cohorts} Within each
cohort and wave, what is the age-adjusted EF network---defined by
partial correlations among working memory, inhibition, and cognitive
flexibility---and what do global strength and expected influence
indicate about overall connectivity and node centrality?

\textbf{RQ2 (Primary): Developmental Change Within Cohorts} RQ2a: Within
each cohort, do the EF conditional associations (partial correlations)
change across adjacent waves (AU: T1→T2→T3; SA: T1→T2), and if so, which
specific edges strengthen or weaken over time?

RQ2b: At the individual level, are gains in one EF component coupled
with gains in other components (i.e., does change in one domain predict
change in another), and does this coupling vary by baseline SES?

\textbf{RQ3 (Secondary): Cross-Cohort Comparisons} At matched waves (T1
and T2), do AU and SA differ in EF conditional associations, and do the
cohorts show divergent developmental trajectories from T1 to T2
(difference-in-differences)?

\textbf{RQ4 (Exploratory): Socioeconomic Moderation of Network
Structure} Within each cohort, does EF network structure differ between
children from high versus low baseline (T1) household income and home
learning environments? Specifically, are edges systematically stronger
or weaker in high-SES groups across waves?

\clearpage

\section{Methods}\label{methods}

\subsection{Participants}\label{participants}

To be added.

\subsection{Measures}\label{measures}

To be added.

\subsection{Statistical Analysis}\label{statistical-analysis}

\subsubsection{Network Estimation}\label{network-estimation}

We estimated EF network structure using Bayesian Gaussian Graphical
Models (BGGM) (\citeproc{ref-williams2020bayesian}{Williams \& Mulder,
2020a}, \citeproc{ref-williams2020bggm}{2020b}) implemented in the BGGM
R package (\citeproc{ref-BGGM}{Williams et al., 2025}). For each
cohort×wave combination, we fit a separate network model to capture the
conditional relationships between the three EF components (inhibition,
cognitive flexibility, and working memory) after controlling for age. We
used default priors and ran MCMC sampling for 5,050 iterations (5,000
posterior samples after 50 burn-in iterations).

\subsubsection{Edge Weights}\label{edge-weights}

Edges represent the conditional associations between pairs of EF
components. Technically, edge weights are partial correlations: the
correlation between two components after controlling for all other
components in the network and age. A positive edge indicates that two EF
components tend to covary (e.g., children with higher working memory
also tend to have higher inhibition, beyond what age would predict),
while a negative edge would indicate an inverse relationship. Edge
weights range from −1 to +1, with values closer to 0 indicating weaker
associations.

For each edge, we report the posterior median (a point estimate of the
edge strength) and the 95\% credible interval (CrI), which contains the
true edge weight with 95\% probability given the data and model. Unlike
frequentist confidence intervals, credible intervals can be directly
interpreted as probability statements about the parameter.

\subsubsection{Meaningfulness Threshold}\label{meaningfulness-threshold}

To distinguish meaningful edges from those that are small or negligible,
we adopted a Region of Practical Equivalence (ROPE) approach
(\citeproc{ref-kruschke2018rejecting}{Kruschke, 2018}). We defined the
ROPE as \textbar r\textbar{} ≤ 0.10, such that edge weights falling
within ±0.10 were considered practically equivalent to zero. For each
edge, we computed P(\textbar r\textbar{} \textgreater{} 0.10): the
posterior probability that the edge weight exceeds the ROPE threshold in
absolute value. Values close to 1 (e.g., P \textgreater{} 0.90) indicate
strong evidence that the edge is meaningfully different from zero, while
values close to 0.5 indicate substantial uncertainty.

This approach differs from traditional null hypothesis significance
testing in two ways. First, it focuses on practical significance (is the
effect large enough to matter?) rather than statistical significance (is
the effect different from exactly zero?). Second, it quantifies evidence
on a continuous scale rather than using a binary reject/fail-to-reject
decision rule.

\subsubsection{Global Strength}\label{global-strength}

Global strength summarizes overall network connectivity by summing the
absolute values of all unique edge weights. Higher global strength
indicates a more densely connected network where EF components are more
strongly coupled. We computed global strength for each posterior sample,
yielding a posterior distribution that captures uncertainty in this
summary metric.

Formally, for a network with \emph{p} nodes, global strength is:

\[
GS = \sum_{i<j} |r_{ij}|
\]

where \emph{r\textsubscript{ij}} is the partial correlation between
nodes \emph{i} and \emph{j}. For our three-node networks, global
strength is simply \textbar r\textsubscript{inhib-cogflex}\textbar{} +
\textbar r\textsubscript{inhib-WM}\textbar{} +
\textbar r\textsubscript{cogflex-WM}\textbar.

\subsubsection{Centrality (Expected
Influence)}\label{centrality-expected-influence}

Expected Influence (\citeproc{ref-robinaugh2016identifying}{Robinaugh et
al., 2016}) quantifies the centrality of each node in the network---that
is, how strongly connected it is to other nodes. Expected influence is
computed as the sum of a node's edge weights to all other nodes (without
taking absolute values), preserving the sign of the relationships. Nodes
with higher expected influence are more central to the network and may
play a more important role in the overall system.

Formally, for node \emph{i}:

\[
EI_i = \sum_{j \neq i} r_{ij}
\]

Unlike older centrality metrics (e.g., betweenness, closeness), expected
influence is appropriate for networks estimated from partial
correlations and does not require thresholding edges or converting the
network to a binary graph.

We chose expected influence over strength (which uses absolute values)
because it preserves directional information: a node can have high
expected influence by being positively connected to many nodes, whereas
strength would treat positive and negative edges equivalently. In EF
networks where edges are predominantly positive, the two metrics
converge, but expected influence provides a more nuanced picture if any
negative edges emerge.

\subsubsection{Change and Comparison
Analyses}\label{change-and-comparison-analyses}

To test for developmental change within cohorts (RQ2), we computed
difference scores between adjacent waves (e.g., Δr = r\textsubscript{T2}
− r\textsubscript{T1}) by subtracting posterior samples pairwise. This
yields a posterior distribution of the change, from which we derived
95\% credible intervals and ROPE probabilities for the difference
(P(\textbar Δr\textbar{} \textgreater{} 0.10)).

For cross-cohort comparisons (RQ3), we computed cohort difference scores
(e.g., r\textsubscript{AU} − r\textsubscript{SA}) at matched waves. To
account for the bounded nature of correlations, we also report
differences on the Fisher z-scale (z = atanh(r)), which transforms
correlations to an unbounded scale where differences are more symmetric.
Results on both scales were substantively similar, indicating our
findings are not artifacts of scale constraints.

For difference-in-differences (DiD) contrasts, we tested whether cohorts
showed divergent developmental trajectories by computing
(Δr\textsubscript{T2-T1})\textsubscript{AU} −
(Δr\textsubscript{T2-T1})\textsubscript{SA}. A meaningful DiD would
indicate that one cohort changed more than the other over time.

\subsubsection{Individual-Level Change
Coupling}\label{individual-level-change-coupling}

To test whether developmental changes across EF components were
coordinated at the individual level (RQ2b), we computed change scores
for each child (ΔEF = EF\textsubscript{T2} − EF\textsubscript{T1}) and
regressed change in one component on change in another, controlling for
baseline performance and age. Significant positive regression
coefficients would indicate coupled development, where gains in one
domain predict gains in another. We tested all pairwise relationships
and examined whether SES moderated coupling strength.

\subsubsection{Moderation Analyses}\label{moderation-analyses}

To test whether socioeconomic factors moderate network structure (RQ4),
we adopted a stratified approach. Within each cohort, we split children
into High vs.~Low groups based on median splits of baseline (T1) income
and home learning environment (HLE), then fit separate BGGM models for
each group at each wave. We computed moderation effects as High − Low
edge differences, with ROPE probabilities indicating the strength of
evidence for moderation (P(\textbar High − Low\textbar{} \textgreater{}
0.10)). This stratified approach is more robust than including SES as a
continuous covariate in a single model, as it allows network structure
to differ nonlinearly across SES groups and does not assume constant
effects across the SES distribution.

\subsubsection{Robustness check: Measurement
Invariance}\label{robustness-check-measurement-invariance}

We tested measurement invariance across cohorts using confirmatory
factor analysis (CFA) in the lavaan R package
(\citeproc{ref-rosseel2012lavaan}{Rosseel, 2012}). We fit a
single-factor model with the three EF components as indicators and
tested three levels of invariance:

\begin{enumerate}
\def\labelenumi{\arabic{enumi}.}
\tightlist
\item
  \textbf{Configural invariance}: The same factor structure (three
  indicators loading on one factor) holds in both cohorts
\item
  \textbf{Metric invariance}: Factor loadings are equal across cohorts
  (components relate to the latent construct equivalently)
\item
  \textbf{Scalar invariance}: Intercepts are equal across cohorts
  (groups have the same baseline levels)
\end{enumerate}

We evaluated invariance using chi-square difference tests (Δχ²) and
changes in fit indices (ΔCFI, ΔRMSEA), with ΔCFI \textless{} 0.01
supporting invariance (\citeproc{ref-cheung2002evaluating}{Cheung \&
Rensvold, 2002}). Metric invariance is a prerequisite for comparing
relationships between variables across groups, while scalar invariance
is required for comparing mean levels.

\subsubsection{Software}\label{software}

All analyses were conducted in R (version 4.5.0). Network models were
fit using BGGM (version 2.1.5), measurement invariance tests used lavaan
(version 0.6.19), and visualizations used ggplot2 (version 4.0.0).
Analysis code is available at {[}include github url{]}.

\clearpage

\section{Results}\label{results}

\textbf{RQ1 (Primary): EF Network Structure Within Cohorts}: Within each
cohort and wave, what is the age-adjusted EF network---defined by
partial correlations among working memory, inhibition, and cognitive
flexibility---and what do global strength and expected influence
indicate about overall connectivity and node centrality?

\subsubsection{Edges}\label{edges}

Within-wave age-adjusted partial correlations are summarized in
Table~\ref{tbl-rq1-gt-all} and visualized in
Figure~\ref{fig-rq1-edge-forest}. In Australia, edges involving working
memory showed the clearest evidence of coupling across waves. The
inhibition--working memory edge was consistently positive (T1: 0.27
{[}0.15, 0.39{]}; T2: 0.32 {[}0.20, 0.44{]}; T3: 0.30 {[}0.12, 0.47{]})
and reliably exceeded the meaningfulness threshold (T1:
P(\textbar r\textbar\textgreater0.1)=1.00; T2:
P(\textbar r\textbar\textgreater0.1)=1.00; T3:
P(\textbar r\textbar\textgreater0.1)=0.98). The cognitive
flexibility--working memory edge was also positive across waves (T1:
0.25 {[}0.13, 0.37{]}; T2: 0.20 {[}0.06, 0.32{]}; T3: 0.13 {[}-0.06,
0.32{]}) and exceeded the meaningfulness threshold (meaningfulness: T1:
P(\textbar r\textbar\textgreater0.1)=0.99; T2:
P(\textbar r\textbar\textgreater0.1)=0.92; T3:
P(\textbar r\textbar\textgreater0.1)=0.63). In contrast, the cognitive
flexibility--inhibition edge was weaker and more wave-dependent (T1:
0.06 {[}-0.07, 0.19{]}; T2: 0.14 {[}0.01, 0.27{]}; T3: 0.19 {[}-0.00,
0.36{]}), with less consistent evidence that it exceeded the
meaningfulness threshold (T1: P(\textbar r\textbar\textgreater0.1)=0.28;
T2: P(\textbar r\textbar\textgreater0.1)=0.75; T3:
P(\textbar r\textbar\textgreater0.1)=0.83).

In South Africa, the inhibition--working memory edge remained positive
across waves (T1: 0.17 {[}0.03, 0.30{]}; T2: 0.19 {[}0.04, 0.33{]}) and
met the meaningfulness criterion more consistently than the other edges
(T1: P(\textbar r\textbar\textgreater0.1)=0.83; T2:
P(\textbar r\textbar\textgreater0.1)=0.89). The cognitive
flexibility--inhibition edge was stronger at T1 (0.21 {[}0.08, 0.34{]})
but attenuated at T2 (0.10 {[}-0.04, 0.24{]}) and did not reliably
exceed the meaningfulness threshold at T2 (T1:
P(\textbar r\textbar\textgreater0.1)=0.95; T2:
P(\textbar r\textbar\textgreater0.1)=0.51). In contrast to the
Australian results, he cognitive flexibility--working memory edge was
comparatively weaker and uncertain across waves (T1: 0.12 {[}-0.02,
0.26{]}; T2: 0.12 {[}-0.03, 0.26{]}).

\clearpage

\begin{table}

{\caption{{Age-adjusted EF network summaries by cohort×wave. Panels show
edges, global strength, and expected influence (posterior median {[}95\%
CrI{]}). For edges, cells also show
\(P(|r| > r\text{ rope})\).}{\label{tbl-rq1-gt-all}}}
\vspace{-20pt}}

\caption*{
{\fontsize{1}{1}\selectfont  \textbackslash{}vspace\{10pt\}\fontsize{8}{10}\selectfont }
} 
\fontsize{8.0pt}{10.0pt}\selectfont
\begin{tabular*}{\linewidth}{@{\extracolsep{\fill}}l|cccccc}
\toprule
Measure & Cohort & Wave & N & \% missing cells & Median [95\% CrI] & P(|r|>0.1) \\ 
\midrule\addlinespace[2.5pt]
\multicolumn{7}{l}{Edges} \\[2.5pt] 
\midrule\addlinespace[2.5pt]
ef\_cogflex - ef\_inhibition & AU & T1 & 232 & 0.1 & 0.06 [-0.07, 0.19] & 0.28 \\ 
ef\_cogflex - ef\_workingmem & AU & T1 & 232 & 0.1 & 0.25 [0.13, 0.37] & 0.99 \\ 
ef\_inhibition - ef\_workingmem & AU & T1 & 232 & 0.1 & 0.27 [0.15, 0.39] & 1.00 \\ 
ef\_cogflex - ef\_inhibition & AU & T2 & 217 & 0.8 & 0.14 [0.01, 0.27] & 0.75 \\ 
ef\_cogflex - ef\_workingmem & AU & T2 & 217 & 0.8 & 0.20 [0.06, 0.32] & 0.92 \\ 
ef\_inhibition - ef\_workingmem & AU & T2 & 217 & 0.8 & 0.32 [0.20, 0.44] & 1.00 \\ 
ef\_cogflex - ef\_inhibition & AU & T3 & 105 & 0.0 & 0.19 [-0.00, 0.36] & 0.83 \\ 
ef\_cogflex - ef\_workingmem & AU & T3 & 105 & 0.0 & 0.13 [-0.06, 0.32] & 0.63 \\ 
ef\_inhibition - ef\_workingmem & AU & T3 & 105 & 0.0 & 0.30 [0.12, 0.47] & 0.98 \\ 
ef\_cogflex - ef\_inhibition & SA & T1 & 217 & 2.0 & 0.21 [0.08, 0.34] & 0.95 \\ 
ef\_cogflex - ef\_workingmem & SA & T1 & 217 & 2.0 & 0.12 [-0.02, 0.26] & 0.63 \\ 
ef\_inhibition - ef\_workingmem & SA & T1 & 217 & 2.0 & 0.17 [0.03, 0.30] & 0.83 \\ 
ef\_cogflex - ef\_inhibition & SA & T2 & 189 & 1.2 & 0.10 [-0.04, 0.24] & 0.51 \\ 
ef\_cogflex - ef\_workingmem & SA & T2 & 189 & 1.2 & 0.12 [-0.03, 0.26] & 0.60 \\ 
ef\_inhibition - ef\_workingmem & SA & T2 & 189 & 1.2 & 0.19 [0.04, 0.33] & 0.89 \\ 
\midrule\addlinespace[2.5pt]
\multicolumn{7}{l}{Global strength} \\[2.5pt] 
\midrule\addlinespace[2.5pt]
Global strength & AU & T1 & 232 & 0.1 & 0.60 [0.44, 0.76] &  \\ 
Global strength & AU & T2 & 217 & 0.8 & 0.66 [0.50, 0.81] &  \\ 
Global strength & AU & T3 & 105 & 0.0 & 0.63 [0.39, 0.84] &  \\ 
Global strength & SA & T1 & 217 & 2.0 & 0.50 [0.32, 0.68] &  \\ 
Global strength & SA & T2 & 189 & 1.2 & 0.42 [0.22, 0.61] &  \\ 
\midrule\addlinespace[2.5pt]
\multicolumn{7}{l}{Expected influence} \\[2.5pt] 
\midrule\addlinespace[2.5pt]
Cognitive flexibility & AU & T1 & 232 & 0.1 & 0.31 [0.16, 0.46] &  \\ 
Inhibition & AU & T1 & 232 & 0.1 & 0.33 [0.18, 0.49] &  \\ 
Working memory & AU & T1 & 232 & 0.1 & 0.52 [0.36, 0.68] &  \\ 
Cognitive flexibility & AU & T2 & 217 & 0.8 & 0.34 [0.19, 0.49] &  \\ 
Inhibition & AU & T2 & 217 & 0.8 & 0.47 [0.30, 0.62] &  \\ 
Working memory & AU & T2 & 217 & 0.8 & 0.52 [0.36, 0.67] &  \\ 
Cognitive flexibility & AU & T3 & 105 & 0.0 & 0.32 [0.10, 0.54] &  \\ 
Inhibition & AU & T3 & 105 & 0.0 & 0.49 [0.26, 0.72] &  \\ 
Working memory & AU & T3 & 105 & 0.0 & 0.44 [0.19, 0.65] &  \\ 
Cognitive flexibility & SA & T1 & 217 & 2.0 & 0.34 [0.17, 0.50] &  \\ 
Inhibition & SA & T1 & 217 & 2.0 & 0.38 [0.21, 0.54] &  \\ 
Working memory & SA & T1 & 217 & 2.0 & 0.29 [0.12, 0.46] &  \\ 
Cognitive flexibility & SA & T2 & 189 & 1.2 & 0.22 [0.04, 0.40] &  \\ 
Inhibition & SA & T2 & 189 & 1.2 & 0.29 [0.10, 0.48] &  \\ 
Working memory & SA & T2 & 189 & 1.2 & 0.31 [0.11, 0.49] &  \\ 
\bottomrule
\end{tabular*}

\end{table}

\begin{figure}

\caption{\label{fig-rq1-edge-forest}\textbf{Within-wave EF coupling
(posterior partial correlations).} Points show posterior medians; bars
show 95\% credible intervals. Solid line = 0; dashed lines = ROPE
(±0.10).}

\centering{

\pandocbounded{\includegraphics[keepaspectratio]{EF_network_analysis_files/figure-pdf/fig-rq1-edge-forest-1.png}}

}

\end{figure}%

\clearpage

\subsubsection{Global strength}\label{global-strength-1}

Global strength estimates (sum of absolute partial correlations over
unique edges) are reported in Table~\ref{tbl-rq1-gt-all} and visualized
over time in Figure~\ref{fig-rq1-global-strength-trend}. Australia
showed stable global strength across waves (T1: 0.60 {[}0.44, 0.76{]};
T2: 0.66 {[}0.50, 0.81{]}; T3: 0.63 {[}0.39, 0.84{]}), whereas South
Africa showed lower global strength and attenuation from T1 to T2 (T1:
0.50 {[}0.32, 0.68{]}; T2: 0.42 {[}0.22, 0.61{]}).

\begin{figure}

\caption{\label{fig-rq1-global-strength-trend}Global strength over time
by cohort. Points show posterior median; whiskers show 95\% credible
intervals. Global strength is the sum of absolute partial correlations
over unique edges.}

\centering{

\pandocbounded{\includegraphics[keepaspectratio]{EF_network_analysis_files/figure-pdf/fig-rq1-global-strength-trend-1.png}}

}

\end{figure}%

\clearpage

\subsubsection{Centrality (Expected
influence)}\label{centrality-expected-influence-1}

Expected influence estimates (signed sum of incident partial
correlations) are summarized in Table~\ref{tbl-rq1-gt-all} and
visualized over time by node in Figure~\ref{fig-rq1-ei-trend-by-node-n}.
In Australia, expected influence was consistently positive across nodes
and waves, with working memory showing the highest expected influence
across waves (T1: 0.52 {[}0.36, 0.68{]}; T2: 0.52 {[}0.36, 0.67{]}; T3:
0.44 {[}0.19, 0.65{]}). Inhibition showed comparatively high expected
influence from T2 onward (T1: 0.33 {[}0.18, 0.49{]}; T2: 0.47 {[}0.30,
0.62{]}; T3: 0.49 {[}0.26, 0.72{]}), whereas cognitive flexibility was
lower and more stable across waves (T1: 0.31 {[}0.16, 0.46{]}; T2: 0.34
{[}0.19, 0.49{]}; T3: 0.32 {[}0.10, 0.54{]}). These patterns are evident
in Figure~\ref{fig-rq1-ei-trend-by-node-n}, which shows sustained
positive centrality for all three nodes in the Australian cohort, with
working memory and inhibition contributing most strongly to within-wave
coupling across time.

In South Africa, expected influence was positive but generally lower,
with attenuation from T1 to T2 for inhibition and cognitive flexibility
(inhibition: T1: 0.38 {[}0.21, 0.54{]}; T2: 0.29 {[}0.10, 0.48{]};
cognitive flexibility: T1: 0.34 {[}0.17, 0.50{]}; T2: 0.22 {[}0.04,
0.40{]}). Working memory showed comparatively stable expected influence
across waves (T1: 0.29 {[}0.12, 0.46{]}; T2: 0.31 {[}0.11, 0.49{]}).
Figure~\ref{fig-rq1-ei-trend-by-node-n} shows a downward shift from T1
to T2 for inhibition and cognitive flexibility in South Africa, while
working memory remains comparatively steady. \clearpage

\begin{figure}

\caption{\label{fig-rq1-ei-trend-by-node-n}Expected influence over time
by cohort and node. Points show posterior median; whiskers show 95\%
credible intervals. N per wave is shown beneath the zero line within
each panel.}

\centering{

\pandocbounded{\includegraphics[keepaspectratio]{EF_network_analysis_files/figure-pdf/fig-rq1-ei-trend-by-node-n-1.png}}

}

\end{figure}%

\clearpage

\textbf{RQ2a (Primary): Developmental Change Within Cohorts}: Within
each cohort, do the EF conditional associations (partial correlations)
change across adjacent waves (AU: T1→T2→T3; SA: T1→T2), and if so, which
specific edges strengthen or weaken over time?

We found little evidence for meaningful change in EF network summaries
across adjacent waves within either cohort (see
Table~\ref{tbl-rq2-gt-all}). Posterior 95\% CrIs for all Δs included 0,
and probabilities that edge changes exceeded the ROPE
(\textbar Δr\textbar{} \textgreater{} 0.10) were modest (≈0.3--0.6),
indicating substantial uncertainty and no clear support for meaningful
change.

\begin{table}

{\caption{{Change in EF network summaries across adjacent waves within
cohort. Cells show posterior median {[}95\% CrI{]} for Δ (later −
earlier). For edges, cells also show
\(P(|\Delta r| > 0.10)\).}{\label{tbl-rq2-gt-all}}}
\vspace{-20pt}}

\caption*{
{\fontsize{1}{1}\selectfont  \textbackslash{}vspace\{10pt\}\fontsize{8}{10}\selectfont }
} 
\fontsize{8.0pt}{10.0pt}\selectfont
\begin{tabular*}{\linewidth}{@{\extracolsep{\fill}}l|cccc}
\toprule
Measure & Cohort & Contrast & Median [95\% CrI] & P(|Δr|>0.1) \\ 
\midrule\addlinespace[2.5pt]
\multicolumn{5}{l}{Edges (Δr)} \\[2.5pt] 
\midrule\addlinespace[2.5pt]
ef\_cogflex - ef\_inhibition & AU & T2 − T1 & 0.08 [-0.10, 0.27] & 0.46 \\ 
ef\_cogflex - ef\_workingmem & AU & T2 − T1 & -0.05 [-0.24, 0.12] & 0.35 \\ 
ef\_inhibition - ef\_workingmem & AU & T2 − T1 & 0.05 [-0.12, 0.22] & 0.32 \\ 
ef\_cogflex - ef\_inhibition & AU & T3 − T2 & 0.04 [-0.19, 0.27] & 0.42 \\ 
ef\_cogflex - ef\_workingmem & AU & T3 − T2 & -0.06 [-0.30, 0.17] & 0.46 \\ 
ef\_inhibition - ef\_workingmem & AU & T3 − T2 & -0.02 [-0.23, 0.19] & 0.36 \\ 
ef\_cogflex - ef\_inhibition & SA & T2 − T1 & -0.11 [-0.30, 0.08] & 0.56 \\ 
ef\_cogflex - ef\_workingmem & SA & T2 − T1 & -0.01 [-0.20, 0.19] & 0.32 \\ 
ef\_inhibition - ef\_workingmem & SA & T2 − T1 & 0.02 [-0.17, 0.22] & 0.34 \\ 
\midrule\addlinespace[2.5pt]
\multicolumn{5}{l}{Global strength (Δ)} \\[2.5pt] 
\midrule\addlinespace[2.5pt]
Global strength & AU & T2 − T1 & 0.07 [-0.16, 0.28] &  \\ 
Global strength & AU & T3 − T2 & -0.03 [-0.32, 0.24] &  \\ 
Global strength & SA & T2 − T1 & -0.09 [-0.35, 0.18] &  \\ 
\midrule\addlinespace[2.5pt]
\multicolumn{5}{l}{Expected influence (Δ)} \\[2.5pt] 
\midrule\addlinespace[2.5pt]
Cognitive flexibility & AU & T2 − T1 & 0.03 [-0.19, 0.25] & 0.37 \\ 
Inhibition & AU & T2 − T1 & 0.13 [-0.09, 0.35] & 0.64 \\ 
Working memory & AU & T2 − T1 & -0.01 [-0.24, 0.22] & 0.38 \\ 
Cognitive flexibility & AU & T3 − T2 & -0.02 [-0.29, 0.25] & 0.48 \\ 
Inhibition & AU & T3 − T2 & 0.02 [-0.27, 0.29] & 0.50 \\ 
Working memory & AU & T3 − T2 & -0.08 [-0.37, 0.20] & 0.55 \\ 
Cognitive flexibility & SA & T2 − T1 & -0.12 [-0.37, 0.13] & 0.60 \\ 
Inhibition & SA & T2 − T1 & -0.09 [-0.34, 0.16] & 0.53 \\ 
Working memory & SA & T2 − T1 & 0.02 [-0.23, 0.27] & 0.44 \\ 
\bottomrule
\end{tabular*}

\end{table}

\clearpage

\textbf{RQ2b: Individual Differences in Developmental Change}: At the
individual level, are gains in one EF component coupled with gains in
other components (i.e., does change in one domain predict change in
another), and does this coupling vary by baseline SES?

While group-level networks showed stability (RQ2a), we tested whether
individual differences in EF developmental change showed coupling---that
is, whether gains in one EF domain predicted gains in another. For each
child with data at both T1 and T2 (Australia: n = 217; South Africa: n =
189), we computed change scores (ΔEF = T2 − T1) and tested whether
change in one component predicted change in another, controlling for
baseline performance and age (see Table~\ref{tbl-rq2b-change-coupling}).

We found little evidence for systematic coupling of developmental
change. In Australia, all relationships between component changes were
near zero (\textbar β\textbar{} ≤ 0.11, all \emph{p} \textgreater{}
.10). Working memory gains did not reliably predict inhibition gains (β
= -0.00 {[}-0.03, 0.02{]}, \emph{p} = 0.815) nor cognitive flexibility
gains (β = -0.01 {[}-0.53, 0.50{]}, \emph{p} = 0.962). Similarly,
inhibition and cognitive flexibility changes did not predict gains in
other domains.

In South Africa, the pattern was similar, with one marginal exception:
working memory gains showed a tentative positive association with
cognitive flexibility gains (β = 0.43 {[}-0.01, 0.87{]}, \emph{p} =
0.060). All other relationships remained near zero (see
Figure~\ref{fig-rq2b-scatter}), with substantial individual variability
in change patterns but no systematic coupling across domains.

These findings indicate that EF components develop independently at the
individual level during this developmental window, with gains in one
domain not systematically predicting gains in others. This lack of
coupling validates the group-level stability finding (RQ2a): the
stability we observed reflects genuine architectural stability rather
than an artifact of averaging heterogeneous individual trajectories. The
independence of developmental change supports models of EF as a set of
related but dissociable capacities that mature through
component-specific rather than domain-general processes during early
childhood.

\begin{table}

{\caption{{Coupling of developmental change across EF domains. Each row
shows whether change in one EF component (predictor) predicts change in
another component (outcome), controlling for baseline performance and
age. Standardized coefficients with 95\%
CIs.}{\label{tbl-rq2b-change-coupling}}}
\vspace{-20pt}}

\caption*{
{\fontsize{1}{1}\selectfont  \textbackslash{}vspace\{10pt\}\fontsize{8}{10}\selectfont }
} 
\fontsize{8.0pt}{10.0pt}\selectfont
\begin{tabular*}{\linewidth}{@{\extracolsep{\fill}}llcc}
\toprule
Predictor & Outcome & Australia\textsuperscript{\textit{1}} & South Africa \\ 
\midrule\addlinespace[2.5pt]
ΔInhibition & ΔCogFlex & 0.03 [-2.62, 2.68] & -0.52 [-2.30, 1.25] \\ 
ΔInhibition & ΔWorking Memory & 0.11 [-0.48, 0.70] & -0.35 [-0.83, 0.13] \\ 
ΔCogFlex & ΔInhibition & 0.00 [-0.00, 0.01] & 0.00 [-0.01, 0.01] \\ 
ΔCogFlex & ΔWorking Memory & 0.00 [-0.02, 0.03] & 0.02 [-0.01, 0.05] \\ 
ΔWorking Memory & ΔInhibition & -0.00 [-0.03, 0.02] & 0.00 [-0.03, 0.03] \\ 
ΔWorking Memory & ΔCogFlex & -0.01 [-0.53, 0.50] & 0.43 [-0.01, 0.87]† \\ 
\bottomrule
\end{tabular*}
\begin{minipage}{\linewidth}
\vspace{.05em}
\textsuperscript{\textit{1}} † p<.10, * p<.05, ** p<.01, *** p<.001. Models control for baseline outcome and age.\\
\end{minipage}

\end{table}

\begin{figure}

\caption{\label{fig-rq2b-scatter}Individual-level coupling of
developmental change. Points show individual children; lines show
regression slopes. Strong positive slopes indicate that gains in one
domain predict gains in another (coupled development).}

\centering{

\pandocbounded{\includegraphics[keepaspectratio]{EF_network_analysis_files/figure-pdf/fig-rq2b-scatter-1.png}}

}

\end{figure}%

\clearpage

\textbf{RQ3 (Secondary): Cross-Cohort Comparisons} At matched waves (T1
and T2), do AU and SA differ in EF conditional associations, and do the
cohorts show divergent developmental trajectories from T1 to T2
(difference-in-differences)?

At T1, posterior medians suggested a more negative CogFlex--Inhibition
difference (AU−SA) and more positive working-memory edges
(AU\textgreater SA), but uncertainty remains because the 95\% credible
intervals for all between-cohort contrasts cross 0 ---i.e., the
posterior still assigns non-trivial probability to both
AU\textgreater SA and SA\textgreater AU. ROPE probabilities were
nonetheless moderate (P(\textbar Δr\textbar\textgreater0.1)=0.53--0.72),
reflecting that the posterior can place substantial mass beyond ±0.1
while still spanning 0, so magnitude may be meaningful even when
direction is not pinned down. At T2, cohort differences were smaller and
mostly positive, but again CrIs overlapped 0. For DiD, two edges showed
little evidence of cohort-divergent change (posteriors centered near 0),
while CogFlex--Inhibition was most suggestive (median ≈0.20;
P(\textbar DiD\textbar\textgreater0.1)=0.78) yet still crossed 0.
Fisher-z results closely matched Δr, indicating the pattern is not an
artifact of the bounded correlation scale.

\begin{table}

{\caption{{Between-cohort differences (edges only) at matched waves (AU
− SA at T1/T2) and difference-in-differences for change from T1 to T2
(ΔAU − ΔSA). Columns show posterior median {[}95\% CrI{]} on the raw
partial-correlation scale (Δr) and Fisher-z scale (Δz = atanh(r)). ROPE
probabilities are computed on Δr:
\(P(|\Delta r| > 0.10)\).}{\label{tbl-rq3-gt-edges-dual}}}
\vspace{-20pt}}

\caption*{
{\fontsize{1}{1}\selectfont  \textbackslash{}vspace\{10pt\}\fontsize{8}{10}\selectfont }
} 
\fontsize{8.0pt}{10.0pt}\selectfont
\begin{tabular*}{\linewidth}{@{\extracolsep{\fill}}l|ccc}
\toprule
 & \multicolumn{2}{c}{Δr (raw partial correlation)} & \multicolumn{1}{c}{Δz (Fisher atanh scale)} \\ 
\cmidrule(lr){2-3} \cmidrule(lr){4-4}
Edge & Median [95\% CrI] & P(|Δr|>0.1) & Median [95\% CrI] \\ 
\midrule\addlinespace[2.5pt]
\multicolumn{4}{l}{Cohort difference (AU − SA) at T1} \\[2.5pt] 
\midrule\addlinespace[2.5pt]
ef\_cogflex - ef\_inhibition & -0.15 [-0.33, 0.03] & 0.72 & -0.16 [-0.34, 0.03] \\ 
ef\_cogflex - ef\_workingmem & 0.13 [-0.06, 0.31] & 0.62 & 0.13 [-0.06, 0.32] \\ 
ef\_inhibition - ef\_workingmem & 0.11 [-0.08, 0.29] & 0.55 & 0.11 [-0.08, 0.31] \\ 
\midrule\addlinespace[2.5pt]
\multicolumn{4}{l}{Cohort difference (AU − SA) at T2} \\[2.5pt] 
\midrule\addlinespace[2.5pt]
ef\_cogflex - ef\_inhibition & 0.04 [-0.15, 0.23] & 0.36 & 0.04 [-0.16, 0.24] \\ 
ef\_cogflex - ef\_workingmem & 0.08 [-0.11, 0.27] & 0.44 & 0.08 [-0.12, 0.28] \\ 
ef\_inhibition - ef\_workingmem & 0.13 [-0.06, 0.32] & 0.63 & 0.14 [-0.06, 0.34] \\ 
\midrule\addlinespace[2.5pt]
\multicolumn{4}{l}{Difference-in-differences (ΔAU − ΔSA)} \\[2.5pt] 
\midrule\addlinespace[2.5pt]
ef\_cogflex — ef\_workingmem & -0.05 [-0.31, 0.22] & 0.49 & -0.05 [-0.33, 0.22] \\ 
ef\_inhibition — ef\_cogflex & 0.20 [-0.08, 0.47] & 0.78 & 0.20 [-0.08, 0.48] \\ 
ef\_inhibition — ef\_workingmem & 0.03 [-0.24, 0.29] & 0.47 & 0.03 [-0.25, 0.31] \\ 
\bottomrule
\end{tabular*}

\end{table}

\clearpage

\textbf{RQ4 (Exploratory): Socioeconomic Moderation of Network
Structure}: Within each cohort, does EF network structure differ between
children from high versus low baseline (T1) household income and home
learning environments? Specifically, are edges systematically stronger
or weaker in high-SES groups across waves?

Exploratory moderation analyses revealed little consistent evidence that
baseline income or HLE systematically altered EF network structure
within cohorts. However, in Australia at T2, the cognitive
flexibility--working memory edge was substantially weaker among children
from high-income families (High−Low: −0.46 {[}−0.80, −0.11{]},
P(\textbar Δ\textbar\textgreater0.1)=0.98), suggesting that
socioeconomic advantage may be associated with more differentiated EF
development. Conversely, HLE showed tentative positive moderation
patterns (stronger coupling in enriched home environments), though
credible intervals crossed zero.

\begin{table}

{\caption{{Within-cohort moderation of EF edges by baseline (T1) income
and HLE. Cells show posterior median {[}95\% CrI{]} for the difference
in edge strength between high vs low moderator groups (High − Low). ROPE
probability indicates P(\textbar difference\textbar{} \textgreater{}
0.10).}{\label{tbl-rq4-moderation}}}
\vspace{-20pt}}

\caption*{
{\fontsize{1}{1}\selectfont  \textbackslash{}vspace\{10pt\}\fontsize{8}{10}\selectfont }
} 
\fontsize{8.0pt}{10.0pt}\selectfont
\begin{tabular*}{\linewidth}{@{\extracolsep{\fill}}l|cccccc}
\toprule
Edge & Cohort & Wave & n (High) & n (Low) & Median [95\% CrI] & P(|Δ|>0.1) \\ 
\midrule\addlinespace[2.5pt]
\multicolumn{7}{l}{Income} \\[2.5pt] 
\midrule\addlinespace[2.5pt]
ef\_cogflex — ef\_workingmem & AU & T1 & 43 & 146 & -0.16 [-0.50, 0.16] & 0.70 \\ 
ef\_inhibition — ef\_cogflex & AU & T1 & 43 & 146 & -0.15 [-0.50, 0.21] & 0.69 \\ 
ef\_inhibition — ef\_workingmem & AU & T1 & 43 & 146 & -0.23 [-0.59, 0.11] & 0.80 \\ 
ef\_cogflex — ef\_workingmem & AU & T2 & 43 & 136 & -0.46 [-0.80, -0.11] & 0.98 \\ 
ef\_inhibition — ef\_cogflex & AU & T2 & 43 & 136 & 0.06 [-0.29, 0.39] & 0.60 \\ 
ef\_inhibition — ef\_workingmem & AU & T2 & 43 & 136 & -0.07 [-0.41, 0.23] & 0.59 \\ 
ef\_cogflex — ef\_workingmem & AU & T3 & 26 & 59 & 0.21 [-0.27, 0.66] & 0.78 \\ 
ef\_inhibition — ef\_cogflex & AU & T3 & 26 & 59 & -0.19 [-0.66, 0.26] & 0.76 \\ 
ef\_inhibition — ef\_workingmem & AU & T3 & 26 & 59 & -0.30 [-0.76, 0.16] & 0.85 \\ 
ef\_cogflex — ef\_workingmem & SA & T1 & 53 & 158 & 0.10 [-0.23, 0.41] & 0.62 \\ 
ef\_inhibition — ef\_cogflex & SA & T1 & 53 & 158 & 0.17 [-0.13, 0.44] & 0.72 \\ 
ef\_inhibition — ef\_workingmem & SA & T1 & 53 & 158 & -0.24 [-0.59, 0.10] & 0.82 \\ 
ef\_cogflex — ef\_workingmem & SA & T2 & 45 & 138 & -0.18 [-0.51, 0.17] & 0.72 \\ 
ef\_inhibition — ef\_cogflex & SA & T2 & 45 & 138 & 0.01 [-0.33, 0.34] & 0.56 \\ 
ef\_inhibition — ef\_workingmem & SA & T2 & 45 & 138 & 0.09 [-0.25, 0.40] & 0.62 \\ 
\midrule\addlinespace[2.5pt]
\multicolumn{7}{l}{HLE} \\[2.5pt] 
\midrule\addlinespace[2.5pt]
ef\_cogflex — ef\_workingmem & AU & T1 & 86 & 103 & -0.01 [-0.30, 0.25] & 0.47 \\ 
ef\_inhibition — ef\_cogflex & AU & T1 & 86 & 103 & 0.11 [-0.17, 0.40] & 0.60 \\ 
ef\_inhibition — ef\_workingmem & AU & T1 & 86 & 103 & -0.05 [-0.33, 0.22] & 0.52 \\ 
ef\_cogflex — ef\_workingmem & AU & T2 & 82 & 97 & 0.24 [-0.04, 0.52] & 0.85 \\ 
ef\_inhibition — ef\_cogflex & AU & T2 & 82 & 97 & 0.11 [-0.18, 0.41] & 0.61 \\ 
ef\_inhibition — ef\_workingmem & AU & T2 & 82 & 97 & -0.01 [-0.28, 0.25] & 0.46 \\ 
ef\_cogflex — ef\_workingmem & AU & T3 & 38 & 47 & 0.20 [-0.24, 0.62] & 0.77 \\ 
ef\_inhibition — ef\_cogflex & AU & T3 & 38 & 47 & 0.30 [-0.11, 0.69] & 0.86 \\ 
ef\_inhibition — ef\_workingmem & AU & T3 & 38 & 47 & 0.02 [-0.39, 0.43] & 0.65 \\ 
ef\_cogflex — ef\_workingmem & SA & T1 & 77 & 134 & 0.02 [-0.27, 0.30] & 0.49 \\ 
ef\_inhibition — ef\_cogflex & SA & T1 & 77 & 134 & -0.01 [-0.28, 0.27] & 0.49 \\ 
ef\_inhibition — ef\_workingmem & SA & T1 & 77 & 134 & -0.03 [-0.33, 0.25] & 0.52 \\ 
ef\_cogflex — ef\_workingmem & SA & T2 & 73 & 110 & 0.28 [-0.02, 0.56] & 0.89 \\ 
ef\_inhibition — ef\_cogflex & SA & T2 & 73 & 110 & 0.10 [-0.21, 0.39] & 0.59 \\ 
ef\_inhibition — ef\_workingmem & SA & T2 & 73 & 110 & 0.04 [-0.25, 0.34] & 0.54 \\ 
\bottomrule
\end{tabular*}

\end{table}

\clearpage

\subsection{Robustness Check: Measurement
Invariance}\label{robustness-check-measurement-invariance-1}

To test whether EF network differences reflected genuine structural
differences versus measurement artifacts, we conducted measurement
invariance tests across cohorts at T1 and T2. At T1, metric invariance
was not supported (Δχ² = 10.85, Δ\emph{df} = 2, \emph{p} =
\textless.01), with model fit deteriorating from CFI = 1.000 to 0.780.
This indicates that the factor loadings---the relationships between EF
components and the underlying construct---differed significantly across
cohorts at baseline, validating our RQ1 finding that network
architectures are context-dependent. However, by T2, metric invariance
was achieved (Δχ² = 1.03, \emph{p} = 0.596; CFI remained at 1.000),
indicating that structural relationships had converged despite
differences in baseline performance levels (scalar invariance: \emph{p}
= \textless.001).

\begin{table}

{\caption{{Measurement invariance tests across cohorts. Models test
whether the three-component EF structure is equivalent between Australia
and South Africa.}{\label{tbl-measurement-invariance}}}
\vspace{-20pt}}

\caption*{
{\fontsize{1}{1}\selectfont  \textbackslash{}vspace\{10pt\}\fontsize{8}{10}\selectfont }
} 
\fontsize{8.0pt}{10.0pt}\selectfont
\begin{tabular*}{\linewidth}{@{\extracolsep{\fill}}llccccccc}
\toprule
{\bfseries Wave} & {\bfseries Model} & {\bfseries χ²} & {\bfseries \emph{df}} & {\bfseries \emph{p}} & {\bfseries CFI}\textsuperscript{\textit{1}} & {\bfseries TLI} & {\bfseries RMSEA [90\% CI]} & {\bfseries SRMR} \\ 
\midrule\addlinespace[2.5pt]
T1 & Configural & 0.00 & 0 & NA & 1.000 & 1.000 & NA [NA, NA] & 0.000 \\ 
T1 & Metric & 20.61 & 2 & <.001 & 0.780 & 0.339 & 0.204 [0.000, 0.000] & 0.050 \\ 
T1 & Scalar & 46.93 & 4 & <.001 & 0.492 & 0.237 & 0.219 [0.163, 0.280] & 0.102 \\ 
T2 & Configural & 0.00 & 0 & NA & 1.000 & 1.000 & NA [NA, NA] & 0.000 \\ 
T2 & Metric & 1.05 & 2 & 0.592 & 1.000 & 1.040 & 0.000 [0.000, 0.121] & 0.016 \\ 
T2 & Scalar & 29.68 & 4 & <.001 & 0.644 & 0.466 & 0.178 [0.110, 0.254] & 0.069 \\ 
\bottomrule
\end{tabular*}
\begin{minipage}{\linewidth}
\vspace{.05em}
\textsuperscript{\textit{1}} Good fit: CFI/TLI > .95, RMSEA < .06, SRMR < .08. Configural = same structure across groups; Metric = equal factor loadings; Scalar = equal loadings + intercepts.\\
\end{minipage}

\end{table}

\begin{table}

{\caption{{Model comparison tests for measurement invariance. Δχ² tests
whether constraining parameters significantly worsens
fit.}{\label{tbl-measurement-invariance-comparisons}}}
\vspace{-20pt}}

\caption*{
{\fontsize{1}{1}\selectfont  \textbackslash{}vspace\{10pt\}\fontsize{8}{10}\selectfont }
} 
\fontsize{8.0pt}{10.0pt}\selectfont
\begin{tabular*}{\linewidth}{@{\extracolsep{\fill}}clcccc}
\toprule
Wave & Comparison & Δχ² & Δ\emph{df} & \emph{p}\textsuperscript{\textit{1}} & Invariance \\ 
\midrule\addlinespace[2.5pt]
T1 & Configural vs Metric & 10.85 & 2 & <.01 & {\cellcolor[HTML]{F08080}{Rejected}} \\ 
T1 & Metric vs Scalar & 29.88 & 2 & <.001 & {\cellcolor[HTML]{F08080}{Rejected}} \\ 
T2 & Configural vs Metric & 1.03 & 2 & 0.596 & {\cellcolor[HTML]{90EE90}{Supported}} \\ 
T2 & Metric vs Scalar & 19.11 & 2 & <.001 & {\cellcolor[HTML]{F08080}{Rejected}} \\ 
\bottomrule
\end{tabular*}
\begin{minipage}{\linewidth}
\vspace{.05em}
\textsuperscript{\textit{1}} Non-significant p-value supports invariance (constraints do not significantly worsen fit).\\
\end{minipage}

\end{table}

\newpage

\section{References}\label{references}

\phantomsection\label{refs}
\begin{CSLReferences}{1}{0}
\bibitem[\citeproctext]{ref-cheung2002evaluating}
Cheung, G. W., \& Rensvold, R. B. (2002). Evaluating goodness-of-fit
indexes for testing measurement invariance. \emph{Structural Equation
Modeling}, \emph{9}(2), 233--255.

\bibitem[\citeproctext]{ref-kruschke2018rejecting}
Kruschke, J. (2018). \emph{Rejecting or accepting parameter values in
bayesian estimation. Advances in methods and practices in psychological
science, 1 (2), 270--280}.

\bibitem[\citeproctext]{ref-robinaugh2016identifying}
Robinaugh, D. J., Millner, A. J., \& McNally, R. J. (2016). Identifying
highly influential nodes in the complicated grief network. \emph{Journal
of Abnormal Psychology}, \emph{125}(6), 747.

\bibitem[\citeproctext]{ref-rosseel2012lavaan}
Rosseel, Y. (2012). Lavaan: An r package for structural equation
modeling. \emph{Journal of Statistical Software}, \emph{48}(1), 1--36.

\bibitem[\citeproctext]{ref-williams2020bayesian}
Williams, D. R., \& Mulder, J. (2020a). Bayesian hypothesis testing for
gaussian graphical models: Conditional independence and order
constraints. \emph{Journal of Mathematical Psychology}, \emph{99},
102441.

\bibitem[\citeproctext]{ref-williams2020bggm}
Williams, D. R., \& Mulder, J. (2020b). BGGM: Bayesian gaussian
graphical models in r. \emph{The Journal of Open Source Software},
\emph{5}(51), 2111.

\bibitem[\citeproctext]{ref-BGGM}
Williams, D. R., Mulder, J., \& Rast, P. (2025). \emph{BGGM: Bayesian
gaussian graphical models} (R package version 2.1.6). The Comprehensive
R Archive Network (CRAN).
\url{https://doi.org/10.32614/CRAN.package.BGGM}

\end{CSLReferences}

\appendix

\section{Demographics}\label{apx-render-table1}

\begin{landscape}

\begin{table}
\caption{Age and executive function by country and timepoint. All summaries are
anchored to children with any EF observed at that timepoint (partial EF
allowed).}\tabularnewline

\fontsize{8.0pt}{10.0pt}\selectfont
\begin{tabular*}{1\linewidth}{@{\extracolsep{\fill}}l|l|ccccc}
\toprule
\multicolumn{2}{c}{} & Australia — T1 & Australia — T2 & Australia — T3 & SA — T1 & SA — T2 \\ 
\midrule\addlinespace[2.5pt]
{\bfseries \multirow[t]{6}{*}{Executive function}} & Cognitive flexibility, Mean (SD) & 4.69 (4.12) & 5.99 (4.01) & 8.76 (2.14) & 7.73 (2.15) & 8.16 (2.73) \\ 
 & Cognitive flexibility, n & 231 & 216 & 106 & 214 & 187 \\ 
 & Inhibition, Mean (SD) & 0.56 (0.19) & 0.70 (0.19) & 0.77 (0.17) & 0.59 (0.24) & 0.73 (0.21) \\ 
 & Inhibition, n & 232 & 215 & 105 & 213 & 188 \\ 
 & Working memory, Mean (SD) & 1.50 (0.95) & 1.82 (0.89) & 2.56 (0.79) & 1.55 (0.81) & 2.00 (0.83) \\ 
 & Working memory, n & 232 & 215 & 105 & 211 & 188 \\ 
\midrule\addlinespace[2.5pt]
{\bfseries \multirow[t]{2}{*}{Age}} & Age (years), Mean (SD) & 4.43 (0.38) & 4.99 (0.38) & 5.92 (0.33) & 4.70 (0.60) & 5.35 (0.57) \\ 
 & Age (years), Median [min, max] & 4.44 [3.20, 5.24] & 4.99 [3.74, 5.88] & 5.90 [5.27, 6.59] & 4.68 [2.80, 5.89] & 5.37 [3.54, 6.55] \\ 
\midrule\addlinespace[2.5pt]
{\bfseries \multirow[t]{3}{*}{Sex}} & Female, n (\%) & 107 (46.1\%) & 101 (46.5\%) & 53 (50.0\%) & 117 (48.1\%) & 87 (35.8\%) \\ 
 & Male, n (\%) & 125 (53.9\%) & 116 (53.5\%) & 53 (50.0\%) & 126 (51.9\%) & 105 (43.2\%) \\ 
 & Missing, n (\%) & NA & NA & NA & NA & 51 (21.0\%) \\ 
\bottomrule
\end{tabular*}
\end{table}

\end{landscape}

\begin{landscape}

\begin{table}
\caption{Socioeconomic status and home learning environment by country and
timepoint.}\tabularnewline

\fontsize{8.0pt}{10.0pt}\selectfont
\begin{tabular*}{1\linewidth}{@{\extracolsep{\fill}}l|l|ccccc}
\toprule
\multicolumn{2}{c}{} & Australia — T1 & Australia — T2 & Australia — T3 & SA — T1 & SA — T2 \\ 
\midrule\addlinespace[2.5pt]
{\bfseries \multirow[t]{6}{*}{Caregiver education (AUS)}} & 1 & 18 (7.8\%) & 17 (7.8\%) & 8 (7.5\%) & NA & NA \\ 
 & 2 & 16 (6.9\%) & 15 (6.9\%) & 6 (5.7\%) & NA & NA \\ 
 & 3 & 61 (26.3\%) & 56 (25.8\%) & 24 (22.6\%) & NA & NA \\ 
 & 4 & 68 (29.3\%) & 65 (30.0\%) & 34 (32.1\%) & NA & NA \\ 
 & 5 & 35 (15.1\%) & 34 (15.7\%) & 16 (15.1\%) & NA & NA \\ 
 & NA & 34 (14.7\%) & 30 (13.8\%) & 18 (17.0\%) & NA & NA \\ 
\midrule\addlinespace[2.5pt]
{\bfseries \multirow[t]{4}{*}{Caregiver education (SA)}} & 1 & NA & NA & NA & 30 (13.8\%) & 30 (15.8\%) \\ 
 & 2 & NA & NA & NA & 5 (2.3\%) & 5 (2.6\%) \\ 
 & 3 & NA & NA & NA & 1 (0.5\%) & 1 (0.5\%) \\ 
 & NA & NA & NA & NA & 181 (83.4\%) & 154 (81.1\%) \\ 
\midrule\addlinespace[2.5pt]
{\bfseries \multirow[t]{4}{*}{Household income (AUS)}} & 1 & 20 (8.6\%) & 19 (8.8\%) & 8 (7.5\%) & NA & NA \\ 
 & 2 & 126 (54.3\%) & 117 (53.9\%) & 51 (48.1\%) & NA & NA \\ 
 & 3 & 43 (18.5\%) & 43 (19.8\%) & 27 (25.5\%) & NA & NA \\ 
 & NA & 43 (18.5\%) & 38 (17.5\%) & 20 (18.9\%) & NA & NA \\ 
\midrule\addlinespace[2.5pt]
{\bfseries \multirow[t]{2}{*}{Household income (SA)}} & Household income (SA), Mean (SD) & NA & NA & NA & 7.81 (3.02) & 7.77 (2.98) \\ 
 & Household income (SA), n & NA & NA & NA & 214 & 188 \\ 
\midrule\addlinespace[2.5pt]
{\bfseries \multirow[t]{2}{*}{Home learning environment (AUS)}} & Home learning environment (AUS; HLE Index), Mean (SD) & 17.84 (10.71) & 17.83 (10.45) & 17.17 (11.35) & NA & NA \\ 
 & Home learning environment (AUS; HLE Index), n & 232 & 217 & 106 & NA & NA \\ 
\midrule\addlinespace[2.5pt]
{\bfseries \multirow[t]{8}{*}{Home learning environment (SA)}} & HLE activity frequency (sum), Mean (SD) & NA & NA & NA & 10.13 (3.06) & 12.48 (3.50) \\ 
 & HLE activity frequency (sum), n & NA & NA & NA & 217 & 187 \\ 
 & HLE: books/toys total, Mean (SD) & NA & NA & NA & 2.53 (1.01) & 2.57 (1.07) \\ 
 & HLE: books/toys total, n & NA & NA & NA & 217 & 187 \\ 
 & HLE: time total, Mean (SD) & NA & NA & NA & 4.00 (1.77) & 4.10 (1.84) \\ 
 & HLE: time total, n & NA & NA & NA & 217 & 187 \\ 
 & HLE: unique caregiver types involved, Mean (SD) & NA & NA & NA & 2.26 (1.14) & 1.62 (0.78) \\ 
 & HLE: unique caregiver types involved, n & NA & NA & NA & 214 & 108 \\ 
\bottomrule
\end{tabular*}
\begin{minipage}{\linewidth}
\vspace{.05em}
\textbf{Note.} Summaries are restricted to children with any EF data at each wave (partial EF allowed). In Australia, caregiver education, household income, and the HLE Index are measured once per child and carried across waves; \emph{n} and summary statistics may still differ across T1--T3 because the EF-available sample varies by wave (attrition/partial EF). In South Africa, caregiver education, income, and HLE measures are recorded by wave and summarised only where collected (later-wave cells are NA if not assessed).\\
\end{minipage}
\end{table}

\end{landscape}

\section{Age associations}\label{apx-render-test-re-test-plot}

\begin{figure}[H]

\caption{Executive Function associations across waves. Note:
Standardised scores (z); colour indicates mean age across available
waves}

{\centering \pandocbounded{\includegraphics[keepaspectratio]{EF_network_analysis_files/figure-pdf/test-re-test-plot-1.png}}

}

\end{figure}%

\section{Executive Function across waves}\label{apx-age-ef-wave}

\begin{figure}[H]

\caption{Executive Function by age across waves (z-scored within EF
task). Note: Points jittered + transparent; lines are linear fits with
95\% CI}

{\centering \pandocbounded{\includegraphics[keepaspectratio]{EF_network_analysis_files/figure-pdf/age-ef-wave-1.png}}

}

\end{figure}%

\section{Executive Function scores over time}\label{apx-mean-trend}

\begin{figure}[H]

\caption{\textbf{Mean EF scores over time (children with repeated EF
data).}\\
\emph{Note.} ``EF observed'' at a wave means at least one EF task score
is non-missing. The analytic sample includes only children meeting the
wave-coverage rule: SA = EF observed at both T1 and T2; AU = EF observed
at T1--T3 (``strict'') or at T1 and T2 (``t12'').''}

{\centering \pandocbounded{\includegraphics[keepaspectratio]{EF_network_analysis_files/figure-pdf/mean-trend-1.png}}

}

\end{figure}%

\section{Executive Function individual
trajectories}\label{apx-spaghetti-plot}

\begin{figure}[H]

\caption{\textbf{Individual EF trajectories over time (children with
repeated EF data).}\\
\emph{Note.} ``EF observed'' at a wave means at least one EF task score
is non-missing. The analytic sample includes only children meeting the
wave-coverage rule: SA = EF observed at both T1 and T2; AU = EF observed
at T1--T3 (``strict'') or at T1 and T2 (``t12''). Mean ± 95\% CI is
overlaid. EF scores are z-scored within task (pooled).}

{\centering \pandocbounded{\includegraphics[keepaspectratio]{EF_network_analysis_files/figure-pdf/spaghetti-plot-1.png}}

}

\end{figure}%






\end{document}
