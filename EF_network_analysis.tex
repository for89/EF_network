\documentclass[
  man,
  floatsintext,
  longtable,
  nolmodern,
  notxfonts,
  notimes,
  colorlinks=true,linkcolor=blue,citecolor=blue,urlcolor=blue]{apa7}

\usepackage{amsmath}
\usepackage{amssymb}




\RequirePackage{longtable}
\RequirePackage{threeparttablex}

\makeatletter
\renewcommand{\paragraph}{\@startsection{paragraph}{4}{\parindent}%
	{0\baselineskip \@plus 0.2ex \@minus 0.2ex}%
	{-.5em}%
	{\normalfont\normalsize\bfseries\typesectitle}}

\renewcommand{\subparagraph}[1]{\@startsection{subparagraph}{5}{0.5em}%
	{0\baselineskip \@plus 0.2ex \@minus 0.2ex}%
	{-\z@\relax}%
	{\normalfont\normalsize\bfseries\itshape\hspace{\parindent}{#1}\textit{\addperi}}{\relax}}
\makeatother




\usepackage{longtable, booktabs, multirow, multicol, colortbl, hhline, caption, array, float, xpatch}
\usepackage{subcaption}


\renewcommand\thesubfigure{\Alph{subfigure}}
\setcounter{topnumber}{2}
\setcounter{bottomnumber}{2}
\setcounter{totalnumber}{4}
\renewcommand{\topfraction}{0.85}
\renewcommand{\bottomfraction}{0.85}
\renewcommand{\textfraction}{0.15}
\renewcommand{\floatpagefraction}{0.7}

\usepackage{tcolorbox}
\tcbuselibrary{listings,theorems, breakable, skins}
\usepackage{fontawesome5}

\definecolor{quarto-callout-color}{HTML}{909090}
\definecolor{quarto-callout-note-color}{HTML}{0758E5}
\definecolor{quarto-callout-important-color}{HTML}{CC1914}
\definecolor{quarto-callout-warning-color}{HTML}{EB9113}
\definecolor{quarto-callout-tip-color}{HTML}{00A047}
\definecolor{quarto-callout-caution-color}{HTML}{FC5300}
\definecolor{quarto-callout-color-frame}{HTML}{ACACAC}
\definecolor{quarto-callout-note-color-frame}{HTML}{4582EC}
\definecolor{quarto-callout-important-color-frame}{HTML}{D9534F}
\definecolor{quarto-callout-warning-color-frame}{HTML}{F0AD4E}
\definecolor{quarto-callout-tip-color-frame}{HTML}{02B875}
\definecolor{quarto-callout-caution-color-frame}{HTML}{FD7E14}

%\newlength\Oldarrayrulewidth
%\newlength\Oldtabcolsep


\usepackage{hyperref}




\providecommand{\tightlist}{%
  \setlength{\itemsep}{0pt}\setlength{\parskip}{0pt}}
\usepackage{longtable,booktabs,array}
\usepackage{calc} % for calculating minipage widths
% Correct order of tables after \paragraph or \subparagraph
\usepackage{etoolbox}
\makeatletter
\patchcmd\longtable{\par}{\if@noskipsec\mbox{}\fi\par}{}{}
\makeatother
% Allow footnotes in longtable head/foot
\IfFileExists{footnotehyper.sty}{\usepackage{footnotehyper}}{\usepackage{footnote}}
\makesavenoteenv{longtable}

\usepackage{graphicx}
\makeatletter
\newsavebox\pandoc@box
\newcommand*\pandocbounded[1]{% scales image to fit in text height/width
  \sbox\pandoc@box{#1}%
  \Gscale@div\@tempa{\textheight}{\dimexpr\ht\pandoc@box+\dp\pandoc@box\relax}%
  \Gscale@div\@tempb{\linewidth}{\wd\pandoc@box}%
  \ifdim\@tempb\p@<\@tempa\p@\let\@tempa\@tempb\fi% select the smaller of both
  \ifdim\@tempa\p@<\p@\scalebox{\@tempa}{\usebox\pandoc@box}%
  \else\usebox{\pandoc@box}%
  \fi%
}
% Set default figure placement to htbp
\def\fps@figure{htbp}
\makeatother




\usepackage[nolongtablepatch]{lineno}
\linenumbers



\usepackage{newtx}

\defaultfontfeatures{Scale=MatchLowercase}
\defaultfontfeatures[\rmfamily]{Ligatures=TeX,Scale=1}





\title{Mapping the Structure of Executive Function in Early Childhood: A
Network Analysis Approach}


\shorttitle{Mapping Executive Function}


\usepackage{etoolbox}









\authorsnames[{1,2},{2},{3},{4},{5},{3},{2},{2}]{Fionnuala
O'Reilly,Jelena Sucevic,Caylee Cook,Justin E. Karr,Philippe
Rast,Catherine Draper*,Steven Howard*,Gaia Scerif*}







\authorsaffiliations{
{Stanford University},{University of Oxford},{University of the
Witwatersrand},{University of Kentucky},{University of California,
Davis}}




\leftheader{O'Reilly, Sucevic, Cook, Karr, Rast, Draper*, Howard* and Scerif*}



\abstract{TO BE ADDED. }

\keywords{executive function}

\authornote{\par{\addORCIDlink{Fionnuala
O'Reilly}{0000-0002-4355-9088}}\par{\addORCIDlink{Jelena
Sucevic}{0000-0001-5091-5434}}\par{\addORCIDlink{Caylee
Cook}{0000-0001-9718-8887}}\par{\addORCIDlink{Justin E.
Karr}{0000-0003-3653-332X}}\par{\addORCIDlink{Philippe
Rast}{0000-0003-3630-6629}}\par{\addORCIDlink{Catherine
Draper*}{0000-0002-2885-437X}}\par{\addORCIDlink{Steven
Howard*}{0000-0002-1258-3210}}\par{\addORCIDlink{Gaia
Scerif*}{0000-0002-6371-8874}} 

\par{   The author has no conflict of interest to declare. This work was
supported by X.   }
\par{Correspondence concerning this article should be addressed
to Fionnuala
O'Reilly, Email: \href{mailto:fionnuala.oreilly@psy.ox.ac.uk}{fionnuala.oreilly@psy.ox.ac.uk}}
}

\makeatletter
\let\endoldlt\endlongtable
\def\endlongtable{
\hline
\endoldlt
}
\makeatother

\urlstyle{same}



\usepackage{setspace}
% \AtBeginEnvironment{longtable}{\singlespacing\footnotesize}
\AtBeginEnvironment{longtable}{\singlespacing\renewcommand*{\arraystretch}{1.15}}
\usepackage{booktabs}
\usepackage{caption}
\usepackage{longtable}
\usepackage{colortbl}
\usepackage{array}
\usepackage{anyfontsize}
\usepackage{multirow}
\makeatletter
\@ifpackageloaded{caption}{}{\usepackage{caption}}
\AtBeginDocument{%
\ifdefined\contentsname
  \renewcommand*\contentsname{Table of contents}
\else
  \newcommand\contentsname{Table of contents}
\fi
\ifdefined\listfigurename
  \renewcommand*\listfigurename{List of Figures}
\else
  \newcommand\listfigurename{List of Figures}
\fi
\ifdefined\listtablename
  \renewcommand*\listtablename{List of Tables}
\else
  \newcommand\listtablename{List of Tables}
\fi
\ifdefined\figurename
  \renewcommand*\figurename{Figure}
\else
  \newcommand\figurename{Figure}
\fi
\ifdefined\tablename
  \renewcommand*\tablename{Table}
\else
  \newcommand\tablename{Table}
\fi
}
\@ifpackageloaded{float}{}{\usepackage{float}}
\floatstyle{ruled}
\@ifundefined{c@chapter}{\newfloat{codelisting}{h}{lop}}{\newfloat{codelisting}{h}{lop}[chapter]}
\floatname{codelisting}{Listing}
\newcommand*\listoflistings{\listof{codelisting}{List of Listings}}
\makeatother
\makeatletter
\usepackage{pdflscape}
\makeatother
\makeatletter
\makeatother
\makeatletter
\@ifpackageloaded{caption}{}{\usepackage{caption}}
\@ifpackageloaded{subcaption}{}{\usepackage{subcaption}}
\makeatother

% From https://tex.stackexchange.com/a/645996/211326
%%% apa7 doesn't want to add appendix section titles in the toc
%%% let's make it do it
\makeatletter
\xpatchcmd{\appendix}
  {\par}
  {\addcontentsline{toc}{section}{\@currentlabelname}\par}
  {}{}
\makeatother

%% Disable longtable counter
%% https://tex.stackexchange.com/a/248395/211326

\usepackage{etoolbox}

\makeatletter
\patchcmd{\LT@caption}
  {\bgroup}
  {\bgroup\global\LTpatch@captiontrue}
  {}{}
\patchcmd{\longtable}
  {\par}
  {\par\global\LTpatch@captionfalse}
  {}{}
\apptocmd{\endlongtable}
  {\ifLTpatch@caption\else\addtocounter{table}{-1}\fi}
  {}{}
\newif\ifLTpatch@caption
\makeatother

\begin{document}

\maketitle




\setlength\LTleft{0pt}

\resetlinenumber[1]



\begin{landscape}

\begin{table}
\caption*{
{\fontsize{20}{25}\selectfont  Table 1. Age and executive function by country and timepoint\fontsize{8}{10}\selectfont } \\ 
{\fontsize{14}{17}\selectfont  All summaries are anchored to children with any EF observed at that timepoint (partial EF allowed).\fontsize{8}{10}\selectfont }
} 
\fontsize{8.0pt}{10.0pt}\selectfont
\begin{tabular*}{1\linewidth}{@{\extracolsep{\fill}}l|l|ccccc}
\toprule
\multicolumn{2}{c}{} & Australia — T1 & Australia — T2 & Australia — T3 & SA — T1 & SA — T2 \\ 
\midrule\addlinespace[2.5pt]
{\bfseries \multirow[t]{6}{*}{Executive function}} & Cognitive flexibility, Mean (SD) & 4.69 (4.12) & 5.99 (4.01) & 8.76 (2.14) & 7.73 (2.15) & 8.16 (2.73) \\ 
 & Cognitive flexibility, n & 231 & 216 & 106 & 214 & 187 \\ 
 & Inhibition, Mean (SD) & 0.56 (0.19) & 0.70 (0.19) & 0.77 (0.17) & 0.59 (0.24) & 0.73 (0.21) \\ 
 & Inhibition, n & 232 & 215 & 105 & 213 & 188 \\ 
 & Working memory, Mean (SD) & 1.50 (0.95) & 1.82 (0.89) & 2.56 (0.79) & 1.55 (0.81) & 2.00 (0.83) \\ 
 & Working memory, n & 232 & 215 & 105 & 211 & 188 \\ 
\midrule\addlinespace[2.5pt]
{\bfseries \multirow[t]{2}{*}{Age}} & Age (years), Mean (SD) & 4.43 (0.38) & 4.99 (0.38) & 5.92 (0.33) & 4.71 (0.59) & 5.35 (0.57) \\ 
 & Age (years), Median [min, max] & 4.44 [3.20, 5.24] & 4.99 [3.74, 5.88] & 5.90 [5.27, 6.59] & 4.68 [2.80, 5.89] & 5.36 [3.54, 6.55] \\ 
\midrule\addlinespace[2.5pt]
{\bfseries \multirow[t]{3}{*}{Sex}} & Female, n (\%) & 107 (46.1\%) & 101 (46.5\%) & 53 (50.0\%) & 103 (47.5\%) & 85 (44.7\%) \\ 
 & Male, n (\%) & 125 (53.9\%) & 116 (53.5\%) & 53 (50.0\%) & 114 (52.5\%) & 104 (54.7\%) \\ 
 & Missing, n (\%) & NA & NA & NA & NA & 1 (0.5\%) \\ 
\bottomrule
\end{tabular*}
\end{table}

\end{landscape}

\begin{landscape}

\begin{table}
\caption*{
{\fontsize{20}{25}\selectfont  Table 2. Socioeconomic status and home learning environment by country and timepoint\fontsize{8}{10}\selectfont }
} 
\fontsize{8.0pt}{10.0pt}\selectfont
\begin{tabular*}{1\linewidth}{@{\extracolsep{\fill}}l|l|ccccc}
\toprule
\multicolumn{2}{c}{} & Australia — T1 & Australia — T2 & Australia — T3 & SA — T1 & SA — T2 \\ 
\midrule\addlinespace[2.5pt]
{\bfseries \multirow[t]{6}{*}{Caregiver education (AUS)}} & 1 & 18 (7.8\%) & 17 (7.8\%) & 8 (7.5\%) & NA & NA \\ 
 & 2 & 16 (6.9\%) & 15 (6.9\%) & 6 (5.7\%) & NA & NA \\ 
 & 3 & 61 (26.3\%) & 56 (25.8\%) & 24 (22.6\%) & NA & NA \\ 
 & 4 & 68 (29.3\%) & 65 (30.0\%) & 34 (32.1\%) & NA & NA \\ 
 & 5 & 35 (15.1\%) & 34 (15.7\%) & 16 (15.1\%) & NA & NA \\ 
 & NA & 34 (14.7\%) & 30 (13.8\%) & 18 (17.0\%) & NA & NA \\ 
\midrule\addlinespace[2.5pt]
{\bfseries \multirow[t]{4}{*}{Caregiver education (SA)}} & 1 & NA & NA & NA & 30 (13.8\%) & 30 (15.8\%) \\ 
 & 2 & NA & NA & NA & 5 (2.3\%) & 5 (2.6\%) \\ 
 & 3 & NA & NA & NA & 1 (0.5\%) & 1 (0.5\%) \\ 
 & NA & NA & NA & NA & 181 (83.4\%) & 154 (81.1\%) \\ 
\midrule\addlinespace[2.5pt]
{\bfseries \multirow[t]{4}{*}{Household income (AUS)}} & 1 & 20 (8.6\%) & 19 (8.8\%) & 8 (7.5\%) & NA & NA \\ 
 & 2 & 126 (54.3\%) & 117 (53.9\%) & 51 (48.1\%) & NA & NA \\ 
 & 3 & 43 (18.5\%) & 43 (19.8\%) & 27 (25.5\%) & NA & NA \\ 
 & NA & 43 (18.5\%) & 38 (17.5\%) & 20 (18.9\%) & NA & NA \\ 
\midrule\addlinespace[2.5pt]
{\bfseries \multirow[t]{2}{*}{Household income (SA)}} & Household income (SA), Mean (SD) & NA & NA & NA & 7.80 (3.00) & 7.69 (3.07) \\ 
 & Household income (SA), n & NA & NA & NA & 217 & 190 \\ 
\midrule\addlinespace[2.5pt]
{\bfseries \multirow[t]{2}{*}{Home learning environment (AUS)}} & Home learning environment (AUS; HLE Index), Mean (SD) & 17.84 (10.71) & 17.83 (10.45) & 17.17 (11.35) & NA & NA \\ 
 & Home learning environment (AUS; HLE Index), n & 232 & 217 & 106 & NA & NA \\ 
\midrule\addlinespace[2.5pt]
{\bfseries \multirow[t]{8}{*}{Home learning environment (SA)}} & HLE activity frequency (sum), Mean (SD) & NA & NA & NA & 10.20 (3.03) & 12.45 (3.54) \\ 
 & HLE activity frequency (sum), n & NA & NA & NA & 214 & 183 \\ 
 & HLE: books/toys total, Mean (SD) & NA & NA & NA & 2.55 (1.00) & 2.59 (1.05) \\ 
 & HLE: books/toys total, n & NA & NA & NA & 215 & 184 \\ 
 & HLE: time total, Mean (SD) & NA & NA & NA & 4.00 (1.77) & 4.11 (1.83) \\ 
 & HLE: time total, n & NA & NA & NA & 217 & 186 \\ 
 & HLE: unique caregiver types involved, Mean (SD) & NA & NA & NA & 2.26 (1.14) & 1.67 (0.74) \\ 
 & HLE: unique caregiver types involved, n & NA & NA & NA & 214 & 105 \\ 
\bottomrule
\end{tabular*}
\begin{minipage}{\linewidth}
\vspace{.05em}
\textbf{Note.} Summaries are restricted to children with any EF data at each wave (partial EF allowed). In Australia, caregiver education, household income, and the HLE Index are measured once per child and carried across waves; \emph{n} and summary statistics may still differ across T1--T3 because the EF-available sample varies by wave (attrition/partial EF). In South Africa, caregiver education, income, and HLE measures are recorded by wave and summarised only where collected (later-wave cells are NA if not assessed).\\
\end{minipage}
\end{table}

\end{landscape}

\begin{figure}[H]

\caption{Executive Function associations across waves. Note:
Standardised scores (z); colour indicates mean age across available
waves}

{\centering \pandocbounded{\includegraphics[keepaspectratio]{EF_network_analysis_files/figure-pdf/test-re-test-plot-1.png}}

}

\end{figure}%

\begin{figure}[H]

\caption{Executive Function by age across waves (z-scored within EF
task). Note: Points jittered + transparent; lines are linear fits with
95\% CI}

{\centering \pandocbounded{\includegraphics[keepaspectratio]{EF_network_analysis_files/figure-pdf/age-ef-wave-1.png}}

}

\end{figure}%

\begin{figure}[H]

\caption{\textbf{Mean EF scores over time (children with repeated EF
data).}\\
\emph{Note.} ``EF observed'' at a wave means at least one EF task score
is non-missing. The analytic sample includes only children meeting the
wave-coverage rule: South Africa = EF observed at both T1 and T2;
Australia = EF observed at T1--T3 (``strict'') or at T1 and T2
(``t12'').''}

{\centering \pandocbounded{\includegraphics[keepaspectratio]{EF_network_analysis_files/figure-pdf/mean-trend-1.png}}

}

\end{figure}%

\begin{figure}[H]

\caption{\textbf{Individual EF trajectories over time (children with
repeated EF data).}\\
\emph{Note.} ``EF observed'' at a wave means at least one EF task score
is non-missing. The analytic sample includes only children meeting the
wave-coverage rule: South Africa = EF observed at both T1 and T2;
Australia = EF observed at T1--T3 (``strict'') or at T1 and T2
(``t12''). Mean ± 95\% CI is overlaid. EF scores are z-scored within
task (pooled).}

{\centering \pandocbounded{\includegraphics[keepaspectratio]{EF_network_analysis_files/figure-pdf/spaghetti-plot-1.png}}

}

\end{figure}%

\clearpage

To characterize within-wave coupling among executive function (EF)
domains, we estimated cohort- and wave-specific Gaussian graphical
models (GGMs) in a Bayesian framework. Networks were fitted separately
for each cohort × wave using three EF nodes---inhibition, cognitive
flexibility, and working memory---with edges interpreted as partial
correlations (associations between two EF domains conditional on the
third). Model estimation was restricted to EF-complete cases within each
wave (all three EF nodes observed).

Networks were estimated using BGGM::estimate for continuous-data GGMs
(iter = 12,000; burn-in = 6,000; 4 chains) on EF node scores that were
residualised for age and sex within each cohort × wave × node and then
z-standardised within cohort × wave. For each cohort--wave model, we
extracted posterior draws of the 3×3 partial-correlation matrix and
summarised each edge using the posterior median and 95\% credible
interval (CrI). To support inference about practical magnitude, we
additionally computed a ROPE-style probability for each edge,
P(\textbar r\textbar{} \textgreater{} 0.10), and defined edges as
``meaningful'' where P(\textbar r\textbar{} \textgreater{} 0.10)
\textgreater{} 0.95.

In Australia, the inhibition--working memory edge was robust across
waves (T1: median r = 0.27, 95\% CrI {[}0.15, 0.39{]},
P(\textbar r\textbar{} \textgreater{} 0.1) = 0.998; T2: median r = 0.31,
95\% CrI {[}0.18, 0.42{]}, P(\textbar r\textbar{} \textgreater{} 0.1) =
1.000; T3: median r = 0.30, 95\% CrI {[}0.12, 0.47{]},
P(\textbar r\textbar{} \textgreater{} 0.1) = 0.985) and met the
predefined meaningfulness criterion at each wave (T1: TRUE; T2: TRUE;
T3: TRUE). The cognitive flexibility--working memory edge was strongest
at T1 (median r = 0.25, 95\% CrI {[}0.13, 0.37{]},
P(\textbar r\textbar{} \textgreater{} 0.1) = 0.993) and remained
positive at later waves (T2: median r = 0.19, 95\% CrI {[}0.06, 0.32{]},
P(\textbar r\textbar{} \textgreater{} 0.1) = 0.910; T3: median r = 0.16,
95\% CrI {[}-0.04, 0.34{]}, P(\textbar r\textbar{} \textgreater{} 0.1) =
0.718), but did not consistently exceed the meaningfulness threshold at
T2--T3. The inhibition--cognitive flexibility edge was comparatively
weaker and more uncertain across waves (T1: median r = 0.05, 95\% CrI
{[}-0.08, 0.18{]}, P(\textbar r\textbar{} \textgreater{} 0.1) = 0.248;
T2: median r = 0.15, 95\% CrI {[}0.01, 0.28{]}, P(\textbar r\textbar{}
\textgreater{} 0.1) = 0.762; T3: median r = 0.18, 95\% CrI {[}-0.01,
0.37{]}, P(\textbar r\textbar{} \textgreater{} 0.1) = 0.810).

In South Africa, posterior medians were generally small-to-moderate and
positive across T1--T2 (T1: median r = 0.19, 95\% CrI {[}0.06, 0.32{]},
P(\textbar r\textbar{} \textgreater{} 0.1) = 0.910, median r = 0.15,
95\% CrI {[}0.02, 0.28{]}, P(\textbar r\textbar{} \textgreater{} 0.1) =
0.769, median r = 0.12, 95\% CrI {[}-0.01, 0.26{]},
P(\textbar r\textbar{} \textgreater{} 0.1) = 0.640; T2: median r = 0.10,
95\% CrI {[}-0.04, 0.24{]}, P(\textbar r\textbar{} \textgreater{} 0.1) =
0.509, median r = 0.19, 95\% CrI {[}0.04, 0.32{]},
P(\textbar r\textbar{} \textgreater{} 0.1) = 0.876, median r = 0.11,
95\% CrI {[}-0.04, 0.25{]}, P(\textbar r\textbar{} \textgreater{} 0.1) =
0.568). However, no South Africa edge met the meaningfulness criterion
(P(\textbar r\textbar{} \textgreater{} 0.10) \textgreater{} 0.95) at
either wave. See Figure~\ref{fig-step1-edges-by-edge} for graphical
representation.

Overall, Step 1 indicates stronger and more consistent within-wave EF
coupling in Australia---driven primarily by stable inhibition--working
memory coupling---whereas South Africa shows weaker and/or less certain
within-wave coupling under the ROPE decision rule.

\begin{figure}

\caption{\label{fig-step1-edges-by-edge}Within-wave EF partial
correlations by edge across time. Points show posterior medians;
whiskers show 95\% credible intervals. Solid line = 0; dashed lines =
ROPE bounds (±0.10). Labels show P(\textbar r\textbar{} \textgreater{}
0.10).}

\centering{

\pandocbounded{\includegraphics[keepaspectratio]{EF_network_analysis_files/figure-pdf/fig-step1-edges-by-edge-1.png}}

}

\end{figure}%

\clearpage

To evaluate whether within-wave EF coupling changed over time within
each cohort (RQ2), we quantified posterior changes in each edge between
adjacent waves, defining Δr = r(tp2) − r(tp1). For each cohort ×
contrast, we summarized Δr using the posterior median and 95\% credible
interval (CrI), and computed a ROPE-style probability
P(\textbar Δr\textbar{} \textgreater{} 0.10); changes were classified as
meaningful where P(\textbar Δr\textbar{} \textgreater{} 0.10)
\textgreater{} 0.95.

In Australia, edge changes from T1 to T2 were modest and uncertain
(inhibition--working memory: Δmedian r = 0.03, 95\% CrI {[}-0.14,
0.21{]}, P(\textbar Δr\textbar{} \textgreater{} 0.1) = 0.294, P(Δr
\textgreater{} 0) = 0.656 (n: 231→213); inhibition--cognitive
flexibility: Δmedian r = 0.10, 95\% CrI {[}-0.09, 0.28{]},
P(\textbar Δr\textbar{} \textgreater{} 0.1) = 0.503, P(Δr \textgreater{}
0) = 0.847 (n: 231→213); cognitive flexibility--working memory: Δmedian
r = -0.06, 95\% CrI {[}-0.24, 0.12{]}, P(\textbar Δr\textbar{}
\textgreater{} 0.1) = 0.381, P(Δr \textgreater{} 0) = 0.249 (n:
231→213)) and none met the predefined meaningful-change criterion
(inhibition--working memory: FALSE; inhibition--cognitive flexibility:
FALSE; cognitive flexibility--working memory: FALSE). From T2 to T3,
changes remained small-to-moderate with credible intervals overlapping
zero (inhibition--working memory: Δmedian r = -0.00, 95\% CrI {[}-0.22,
0.20{]}, P(\textbar Δr\textbar{} \textgreater{} 0.1) = 0.365, P(Δr
\textgreater{} 0) = 0.482 (n: 213→105); inhibition--cognitive
flexibility: Δmedian r = 0.03, 95\% CrI {[}-0.20, 0.26{]},
P(\textbar Δr\textbar{} \textgreater{} 0.1) = 0.411, P(Δr \textgreater{}
0) = 0.615 (n: 213→105); cognitive flexibility--working memory: Δmedian
r = -0.04, 95\% CrI {[}-0.27, 0.19{]}, P(\textbar Δr\textbar{}
\textgreater{} 0.1) = 0.414, P(Δr \textgreater{} 0) = 0.377 (n:
213→105)), and again no edge exceeded the meaningful-change threshold
(inhibition--working memory: FALSE; inhibition--cognitive flexibility:
FALSE; cognitive flexibility--working memory: FALSE).

In South Africa (T2-T1), posterior changes were similarly small and
uncertain (inhibition--working memory: Δmedian r = 0.03, 95\% CrI
{[}-0.16, 0.23{]}, P(\textbar Δr\textbar{} \textgreater{} 0.1) = 0.340,
P(Δr \textgreater{} 0) = 0.631 (n: 205→185); inhibition--cognitive
flexibility: Δmedian r = -0.09, 95\% CrI {[}-0.29, 0.11{]},
P(\textbar Δr\textbar{} \textgreater{} 0.1) = 0.493, P(Δr \textgreater{}
0) = 0.178 (n: 205→185); cognitive flexibility--working memory: Δmedian
r = -0.01, 95\% CrI {[}-0.21, 0.19{]}, P(\textbar Δr\textbar{}
\textgreater{} 0.1) = 0.341, P(Δr \textgreater{} 0) = 0.452 (n:
205→185)) and none met the meaningful-change criterion
(inhibition--working memory: FALSE; inhibition--cognitive flexibility:
FALSE; cognitive flexibility--working memory: FALSE). Overall, RQ2
provides little evidence for practically meaningful within-cohort
changes in EF coupling across waves under the ROPE decision rule. Refer
to Figure~\ref{fig-step2-delta-edges} for graphical representation.

\begin{figure}

\caption{\label{fig-step2-delta-edges}RQ2: Change in within-wave EF
coupling across adjacent waves. Points show posterior median Δr (tp2 −
tp1); whiskers show 95\% credible intervals. Solid line = no change
(Δr=0); dashed lines = ROPE bounds (±0.10). Labels show
P(\textbar Δr\textbar{} \textgreater{} 0.10).}

\centering{

\pandocbounded{\includegraphics[keepaspectratio]{EF_network_analysis_files/figure-pdf/fig-step2-delta-edges-1.png}}

}

\end{figure}%

\clearpage

At T1, AU--SA differences were: inhibition--working memory (Δmedian r =
0.12, 95\% CrI {[}-0.06, 0.30{]}, P(\textbar Δr\textbar{} \textgreater{}
0.1) = 0.595, P(Δr \textgreater{} 0) = 0.906), inhibition--cognitive
flexibility (Δmedian r = -0.14, 95\% CrI {[}-0.32, 0.05{]},
P(\textbar Δr\textbar{} \textgreater{} 0.1) = 0.660, P(Δr \textgreater{}
0) = 0.070), and cognitive flexibility--working memory (Δmedian r =
0.13, 95\% CrI {[}-0.05, 0.31{]}, P(\textbar Δr\textbar{} \textgreater{}
0.1) = 0.629, P(Δr \textgreater{} 0) = 0.918).

At T2, AU--SA differences were: inhibition--working memory (Δmedian r =
0.12, 95\% CrI {[}-0.07, 0.31{]}, P(\textbar Δr\textbar{} \textgreater{}
0.1) = 0.599, P(Δr \textgreater{} 0) = 0.897), inhibition--cognitive
flexibility (Δmedian r = 0.05, 95\% CrI {[}-0.15, 0.25{]},
P(\textbar Δr\textbar{} \textgreater{} 0.1) = 0.378, P(Δr \textgreater{}
0) = 0.686), and cognitive flexibility--working memory (Δmedian r =
0.08, 95\% CrI {[}-0.12, 0.27{]}, P(\textbar Δr\textbar{} \textgreater{}
0.1) = 0.455, P(Δr \textgreater{} 0) = 0.788).






\end{document}
